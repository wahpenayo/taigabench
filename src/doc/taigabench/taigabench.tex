% !TEX TS-program = arara
% arara: xelatex: { synctex: on, options: '-halt-on-error' } 
% arara: biber
% arara: makeindex
% arara: xelatex: { synctex: on, options: '-halt-on-error'  } 
% arara: xelatex: { synctex: on, options: '-halt-on-error'   } 
% arara: clean: { files: [ taigabench.aux, taigabench.bbl] }
% arara: clean: { files: [ taigabench.bcf, taigabench.blg] }
% arara: clean: { files: [ taigabench.glg, taigabench.glo] }
% arara: clean: { files: [ taigabench.gls, taigabench.gls] }
% arara: clean: { files: [ taigabench.idx, taigabench.ilg] }
% arara: clean: { files: [ taigabench.ind, taigabench.loe] }
% arara: clean: { files: [ taigabench.lof, taigabench.log ] }
% arara: clean: { files: [ taigabench.log, taigabench.lol ] }
% arara: clean: { files: [ taigabench.out ] }
% arara: clean: { files: [ taigabench.run.xml] }
% arara: clean: { files: [ taigabench.toc, taigabench.xdy] }
% arara: clean: { files: [ taigabench.synctex.gz] }
%-------------------------------------------------------------------------------
\documentclass[10pt]{article}
\input{head}
\input{portrait-1col-header}
\input{clj-listings}
\addbibresource{clojure.bib}
\addbibresource{tex.bib}
%-------------------------------------------------------------------------------
\title{Random forest benchmarks}
\author{\textsc{John Alan McDonald} (wahpenayo at gmail.com)}
\date{\today}
%-------------------------------------------------------------------------------
\begin{document}

\maketitle

%\frontmatter

%\begingroup
%\let\onecolumn\twocolumn
%\sffamily
%\tableofcontents
%\rmfamily
%\endgroup

%\mainmatter

\def\dollar{\text{\$}}    

\section{Introduction}
The document compares several implementations of random forests,
running on a single CPU (ie not distributed).
The main point is to compare 
\href{https://github.com/wahpenayo/taiga/tree/master}{Taiga},
a Clojure library, with open source libraries in
R, Python, C++, and Java.

The high level conclusion is that Taiga is either as fast or as
accurate, and sometimes both, as any of the other libraries tested.

\href{https://xgboost.readthedocs.io/en/latest/}{XGBoost}, a C++
library, consistently suffers in accuracy.

\href{http://www.h2o.ai/}{$\mathrm{H_{2}0.ai}$}, a Java library, is
close to Taiga in both accuracy and performance on some examples,
winning on speed and losing on accuracy in others.

The original R+C 
\href{https://cran.r-project.org/web/packages/randomForest/randomForest.pdf}{randomForest}
library, based on Breiman and Cutler's Fortran implementation,
suffers in both speed and accuracy on small data sets,
and runs out of memory on moderate sized ones (eg $>10^5$ cases in the
example in section~\ref{sec:ontime}).

\href{http://scikit-learn.org/stable/}{scikit-learn}, a widely used
Python machine learning library, has a random forest implementation.
scikit-learn's random forest is comparable to Taiga in speed, 
but signioficantly worse in accuracy, and, like the R randomForest, runs
out of memory on moderate sized datasets (eg $>10^5$ cases in the
example in section~\ref{sec:ontime}).

\href{https://cran.r-project.org/web/packages/randomForestSRC/index.html}{randomForestSRC},
so far only tested on a vector-valued regression example, has the same accuracy as Taiga, but takes
$4-5$ times as long to run.

The code for the benchmarks, as well as this document, can be found in the
\href{https://github.com/wahpenayo/taigabench/tree/master}{TaigaBench}
 library.

The results in this version of the document were run on a
Dell Precision $\mathrm{M}6800$ laptop with a Intel Core
$\mathrm{i}1-4900\mathrm{MQ}$ CPU, an NVIDIA Quadro
$\mathrm{K}5100\mathrm{M}$ GPU, and $32$ GB RAM,
running Windows $7$.

The Taiga Clojure code ran in Java HotSpot(TM) 64-Bit Server VM (build
25.121-b13, mixed mode), with:
\begin{verbatim}
set GC=-XX:+AggressiveHeap -XX:+UseStringDeduplication 
set THRUPUT=-d64 -server -XX:+AggressiveOpts 
:: Leave a couple gb for Windows, Xmx about 2 times Xmn
set XMX=-Xms24g -Xmx24g -Xmn10g 
set CP=-cp ./src/scripts/clojure;lib/*
set JAVA="%JAVA_HOME%\bin\java"
set CMD=%JAVA% %THRUPUT% -ea %GC% %XMX% %CP% clojure.main %*
\end{verbatim}

R libraries ran in Rx$64$ version $3.3.2$. 

XGboost and $\mathrm{H_{2}0.ai}$ were called from their corresponding R
interfaces.

scikit-learn was version $0.18.1$, run in Python $3.5.2$, Anaconda
custom $64$-bit.

\section{\label{sec:ontime}Will a flight will arrive on time?}

\begin{figure}[H]
\noindent \begin{centering}
\includegraphics[width=14cm]{ontime/traintime.png}
\par\end{centering}
\protect\caption{\label{fig:ontime-traintime}Training time in seconds.}
\end{figure}

\begin{figure}[H]
\noindent \begin{centering}
\includegraphics[width=14cm]{ontime/auc.png}
\par\end{centering}
\protect\caption{\label{fig:ontime-auc}Area under the ROC curve.}
\end{figure}


This section uses 
\href{http://stat-computing.org/dataexpo/2009/}
{public data on airline on-time performance} 
from the Data Expo poster session at the
$2009$ Joint Statistical Meetings.
The non-Taiga benchmarking code is inspired by Szilard Pafka's
\href{http://datascience.la/benchmarking-random-forest-implementations/}
{Benchmarking random forest implementations}.
(See also the
\href{https://www.r-bloggers.com/benchmarking-random-forest-implementations/}
{R-Bloggers version}
and the \href{https://github.com/szilard/benchm-ml}
{original benchmarking code}).

%\glsaddallunused
%-------------------------------------------------------------------------------
\appendix
%\part{Appendices}
\input{typesetting}
%-------------------------------------------------------------------------------
%\backmatter

%\part{Backmatter}
%------------------------------------------------------------------------------
% \begingroup  % Temporarily disable \clearpage to show both lists on one page
%   %\let\clearpage\relax    % http://tex.stackexchange.com/a/14511/104449
%   \renewcommand{\listtheoremname}{List of definitions}
%   \textsf{\listoftheorems[ignoreall, show={definition}]}
% \endgroup
%-------------------------------------------------------------------------------
\renewcommand{\listfigurename}{Figures}
\addcontentsline{toc}{chapter}{\listfigurename}
\begingroup
\let\onecolumn\twocolumn
\sffamily
\listoffigures
\rmfamily
\endgroup
%-------------------------------------------------------------------------------
\renewcommand{\lstlistlistingname}{Code samples}
\addcontentsline{toc}{chapter}{\lstlistlistingname}
\begingroup
\let\onecolumn\twocolumn
\sffamily
\lstlistoflistings
\rmfamily
\endgroup
%-------------------------------------------------------------------------------
\printbibliography[heading=bibintoc, title={References}]
%-------------------------------------------------------------------------------

%-------------------------------------------------------------------------------
\end{document}
